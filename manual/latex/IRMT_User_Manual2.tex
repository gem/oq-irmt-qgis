%%%%%%%%%%%%%%%%%%%%%%%%%%%%%%%%%%%%%%%%%
% The Legrand Orange Book
% LaTeX Template
% Version 1.4 (12/4/14)
%
% This template has been downloaded from:
% http://www.LaTeXTemplates.com
%
% Original author:
% Mathias Legrand (legrand.mathias@gmail.com)
%
% License:
% CC BY-NC-SA 3.0 (http://creativecommons.org/licenses/by-nc-sa/3.0/)
%
% Compiling this template:
% This template uses biber for its bibliography and makeindex for its index.
% When you first open the template, compile it from the command line with the 
% commands below to make sure your LaTeX distribution is configured correctly:
%
% 1) pdflatex oqhbt
% 2) makeindex oqhbt.idx -s StyleInd.ist
% 3) biber oqhbt
% 4  makeglossaries oqhbt
% 4) pdflatex oqhbt x 2
%
% After this, when you wish to update the bibliography/index use the appropriate
% command above and make sure to compile with pdflatex several times 
% afterwards to propagate your changes to the document.
%
% This template also uses a number of packages which may need to be
% updated to the newest versions for the template to compile. It is strongly
% recommended you update your LaTeX distribution if you have any
% compilation errors.
%
% Important note:
% Chapter heading images should have a 2:1 width:height ratio,
% e.g. 920px width and 460px height.
%
%%%%%%%%%%%%%%%%%%%%%%%%%%%%%%%%%%%%%%%%%

%----------------------------------------------------------------------------------------
%	PACKAGES AND OTHER DOCUMENT CONFIGURATIONS
%----------------------------------------------------------------------------------------

\documentclass[11pt,fleqn]{book} % Default font size and left-justified equations

\usepackage[top=3cm,bottom=3cm,left=3.2cm,right=3.2cm,headsep=10pt,a4paper]{geometry} % Page margins

\usepackage{xcolor} % Required for specifying colors by name
\definecolor{ocre}{RGB}{243,102,25} % Define the orange color used for highlighting throughout the book
\usepackage{setspace}

% Font Settings
\usepackage{avant} % Use the Avantgarde font for headings
%\usepackage{times} % Use the Times font for headings
\usepackage{mathptmx} % Use the Adobe Times Roman as the default text font together with math symbols from the Sym­bol, Chancery and Com­puter Modern fonts

\usepackage{microtype} % Slightly tweak font spacing for aesthetics
\usepackage[utf8]{inputenc} % Required for including letters with accents
\usepackage[T1]{fontenc} % Use 8-bit encoding that has 256 glyphs

% % Bibliography
% \usepackage{csquotes}
% \usepackage[style=alphabetic,
%             sorting=nyt,
%             sortcites=true,
%             natbib=true,
%             style=authoryear,
%             maxcitenames=2,
%             maxbibnames=100,
%             autopunct=true,
%             babel=hyphen,
%             hyperref=true,
%             doi=true,
%             abbreviate=false,
%             backref=true,
%             uniquename=false,
%             uniquelist=false,
%             backend=biber]{biblatex}
% \addbibresource{./bibliography/hazard.bib} % BibTeX bibliography file
% \defbibheading{bibempty}{}

% Figure caption settings
\usepackage[textfont=it,margin=10pt,font=small,labelfont=bf,labelsep=endash]{caption}
\usepackage{subcaption}
\usepackage{rotating}

% Table
\usepackage{color, colortbl}
\definecolor{almond}{rgb}{0.94, 0.87, 0.8}
\definecolor{ashgrey}{rgb}{0.7, 0.75, 0.71}
\definecolor{anti-flashwhite}{rgb}{0.95, 0.95, 0.96}

% % Index
% \usepackage{calc} % For simpler calculation - used for spacing the index letter headings correctly
% \usepackage{makeidx} % Required to make an index
% \makeindex % Tells LaTeX to create the files required for indexing

\usepackage{todonotes}

%
% Package to create a glossary - It must be uploaded after hyperref
% to produce the glossary: makeglossaries OQB
% \usepackage[acronym,nonumberlist,style=altlist]{glossaries}
% \glstoctrue
% \makeglossaries

% package for bold symbols
\usepackage{bm}

%----------------------------------------------------------------------------------------
% Insert the commands.tex file which contains the majority of the structure 
% behind the template
\input{structure} 


\begin{document}
\lstset{language=Python} % For listings environment - use python
% - - - - - - - - - - - - - - - - - - - - - - - - - - - - - -  Load the glossary
%\input{./book/glossary.tex}

%----------------------------------------------------------------------------------------
%	TITLE PAGE
%----------------------------------------------------------------------------------------

\begingroup
\thispagestyle{empty}
%\AddToShipoutPicture*{\put(6,5){\includegraphics[scale=1]{background}}} % Image background
\par\normalfont\fontsize{15}{15}\sffamily\selectfont
“OpenQuake: Calculate, share, explore”
\centering
\vspace*{9cm}
\par\normalfont\fontsize{35}{35}\sffamily\selectfont
Integrated Risk Modelling Toolkit - User Guide\par % Book title
\endgroup

%----------------------------------------------------------------------------------------
%	COPYRIGHT PAGE
%----------------------------------------------------------------------------------------

\newpage
~\vfill
\thispagestyle{empty}

\noindent Copyright \copyright\ 2014 GEM Foundation\\ % Copyright notice

\noindent \textsc{Published by GEM Foundation}\\ % Publisher

\noindent \textsc{globalquakemodel.org/openquake}\\ % URL

\noindent 
   {\textbf{Citation}} \hfill \\
   Please cite this document as:\\
   Burton, C. and Tormene, P. (2015) OpenQuake Integrated Risk Modelling
   Toolkit - User Guide. \textit{Global Earthquake Model (GEM). Technical
   Report}\\
   
   {\bf{Disclaimer}} \hfill \\
\noindent
   The ``Integrated Risk Modelling Tookit - User Guide'' is distributed in the hope that it will be useful, but without any warranty: without 
   even the implied warranty of merchantability or fitness for a 
   particular purpose. While every 
   precaution has been taken in the preparation of this document, in 
   no event shall the authors of the manual and the GEM Foundation be 
   liable to any party for direct, indirect, special, incidental, or 
   consequential damages, including lost profits, arising out of the 
   use of information contained in this document or from the use of 
   programs and source code that may accompany it, even if the authors 
   and GEM Foundation have been advised of the possibility of such damage. 
   The Book provided hereunder is on as "as is" basis, and the authors 
   and GEM Foundation have no obligations to provide maintenance, support,
   updates, enhancements, or modifications. 
   \hfill \\
   The current version of the book has been revised only by members of 
   the GEM model facility and it must be considered a draft copy. 
   %
   \vspace{0.4cm} \hfill \\
   {\bf{License}} \hfill \\
   This Book is distributed under the Creative Common License 
   Attribution-NonCommercial-NoDerivs 3.0 Unported (CC BY-NC-ND 3.0) 
   (see link below). You can download this Book and share it with 
   others as long as you provide proper credit, but you cannot change 
   it in any way or use it commercially. 
   \hfill \\

\noindent \textit{First printing, August 2015} % Printing/edition date

%----------------------------------------------------------------------------------------
%	TABLE OF CONTENTS
%----------------------------------------------------------------------------------------

% \chapterimage{../images/cover.pdf} % Table of contents heading image

\pagestyle{empty} % No headers

\tableofcontents % Print the table of contents itself

\cleardoublepage % Forces the first chapter to start on an odd page so it's on the right

\pagestyle{fancy} % Print headers again

% Include chapters
\chapter{Introduction}
At the core of the Global Earthquake Model (GEM) is the development of
state-of-the-art modeling capabilities and a suite of software tools that can
be utilized worldwide for the assessment and communication of earthquake risk.
For a more holistic assessment of the scale and consequences of earthquake
impacts, a set of methods, metrics, and tools are incorporated into the GEM
modelling framework to assess earthquake impact potential beyond direct
physical impacts and loss of life. This is because with increased exposure of
people, livelihoods, and property to earthquakes, the potential for social and
economic impacts of earthquakes cannot be ignored. Not only is it vital to
evaluate and benchmark the conditions within social systems that lead to
adverse earthquake impacts and loss, it is equally important to measure the
capacity of populations to respond to damaging events and to provide a set of
metrics for priority setting and decision-making.

The employment  of a methodology and workflow necessary for the evaluation of
seismic risk that is integrated and holistic begins with the Integrated Risk
Modelling Toolkit (IRMT). The IRMT is QGIS plugin that was developed by the
\href{www.globalquakemodel.org}{Global Earthquake Model (GEM) Foundation} and
co-designed by GEM and the \href{www.cedim.de/english/index.php}{Center for
Disaster Management and Risk Reduction Technology (CEDIM)}. The plugin allows
users to form an integrated workflow for the construction of metrics used to
assess characteristics within societies that affect earthquake risk by
providing a GIS-based platform for the construction of indicators and composite
indices to foster comparative assessments. Here, an indicator is defined as a
piece of information that summarizes the characteristics of a system or
highlights what is happening in a system. An indicator is a quantitative or
qualitative measure derived from observed facts that simplify and communicate
the reality of a complex situation. Indicators reveal the relative position of
the phenomena being measured and when evaluated over time, can illustrate the
magnitude of change (a little or a lot) as well as direction of change (up or
down; increasing or decreasing). The mathematical combination (or aggregation
as it is termed) of a set of indicators forms a composite indicator (or
composite index or indices).

As part of the workflow, the IRMT facilitates the integration of composite
indicators of socio-economic characteristics with measures of physical risk
(i.e.\ estimations of human or economic loss) from the OpenQuake Engine
(OQ-engine)\footnote{Silva, V., Crowley, H., Pagani, M., Monelli, D., and
Pinho, R., 2014. Development of the OpenQuake engine, the Global Earthquake
Model's open-source software for seismic risk assessment. Natural Hazards
72(3), 1409-1427}, or other sources, to form what is referred to as an
integrated risk assessment. Although the tool may be utilized for any type of
indicator development, it is encouraged that composite indicators of social
vulnerability are developed within this integrated risk framework. Social
vulnerability is defined here as characteristics or qualities within social
systems that create the potential for harm or loss from damaging hazard events.
Given equal exposure to natural threats, such as an earthquake, two groups may
vary in their social vulnerability due to their pre-existing social
characteristics, where differences according to wealth, gender, race, class,
history, and sociopolitical organization influence the patterns of loss,
mortality, and the ability to reconstruct following damaging events.

The focus on the development of indicators of social vulnerability, and
ultimately integrated risk, will allow researchers, decision-makers, and other
relevant stakeholders to:

\begin{itemize}
    \item consider loss and damage as part of a dynamic system in which
        interactions between natural systems and societal factors redistribute
        risk before an event and redistribute loss after an event
    \item mainstream socio-economic vulnerability and resilience in earthquake
        loss and damage policy discussions
    \item evaluate loss and damage taking social factors into account at
        different time and space scales
    \item use risk assessments in benchmarking exercises to monitor trends in
        earthquake risk over time
    \item recognize that both causes and solutions for earthquake loss are
        found in human, environmental, and built-environmental interactions
    \item help decision-makers develop a common dialog that pertains to the
        factors that they should concentrate on to reduce risk and strengthen
        resilience.
\end{itemize}

The development of composite indicators is not new to research fields and
occupations requiring empirical measurement, and a vast literature on composite
indicators exists that outline methodological approaches for index construction
and validation. To accompany this manual we suggest the use of two popular
resources aimed at providing a guide for the construction and use of composite
indicators.

\begin{enumerate}
    \item Nardo, M., Saisana, M., Saltelli, A. and Tarantola, S. 2005. Tools
        for composite indicators Building. Ispara, Italy: Joint Research Center
        of the European Commission.
    \item Nardo, M., Saisana, M., Saltelli, A. and Tarantola, S. 2008. Handbook
        on constructing composite indicators: Methodology and user guide.
        Paris, France: OECD Publishing.
\end{enumerate}

This literature outlines the process of robust composite indicator construction
that contains a number of steps. The IRMT leverages the QGIS platform to guide
the user through the major steps for index construction. These steps include 1)
the selection of variables; 2) data normalization/standardization; 3) weighting
and aggregation to produce composite indicators; 4) risk integration using
OpenQuake risk estimates; and 5) the presentation of the results. Brief
descriptions of the tool's components and the workflow to develop integrated
risk models are outlined in the sections below.

\include{02_definitions}
\section{OpenQuake Platform connection settings}

\begin{figure}
    \centering
    \includegraphics[width=0.8\textwidth]{../images/image07}
    \caption{OQ-Platform connection settings}
    \label{fig:connection_settings}
\end{figure}

Some of the functionalities provided by the plugin, such as the ability to work
with GEM data, require the interaction between the plugin itself and the
OpenQuake Platform (OQ-Platform). The OQ-Platform is a web-based portal to
visualize, explore and share GEM's datasets, tools and models. In the “Platform
Settings” dialog displayed in Figure~\ref{fig:connection_settings}, credentials
must be inserted to authenticate the user and to allow the user to log into the
OQ-Platform. In the ‘Host' field insert the URL of GEM's production
installation of the \href{https://platform.openquake.org}{OQ-Platform} or a
different installation if you have URL access. If you still haven't signed up
to the OQ-Platform, you can do so by clicking `Register to the OQ-Platform'.
This will open a new web browser and
\href{https://platform.openquake.org/account/signup/}{sign up page}.  The
checkbox labeled `Developer mode (requires restart)' can be used to increase
the verbosity of logging. The latter is useful for developers or advanced users
because logging is critical for troubleshooting, but it is not recommended for
standard users.

\include{04_load_indicators_from_platform}
\chapter{Download project from the OpenQuake Platform}

\begin{figure}
    \centering
    \includegraphics[width=\textwidth]{../images/image15}
    \caption{Download project from the OpenQuake Platform}
    \label{fig:download_project_from_platform}
\end{figure}

An additional option to access data is by downloading projects shared by others
on the OQ-Platform. By clicking the `Download project from the OpenQuake
Platform', the above dialog is opened (Figure 4). Here, a list of available
projects is displayed. The list will contain the titles of projects for which
the user has been granted editing privileges (their own projects or those
shared with them by other users). When a project is selected from the list, its
title, abstract, bounding box and keywords are displayed in the lower textbox
that is utilized to delineate important attributes of the project's definition.
The label directly above the textbox displays an ID that uniquely identifies
the layer used in the OpenQuake-platform.

By pressing `OK', the layer will be downloaded into the QGIS\@. If the associated
project only contains one `project definition', it will be automatically be
selected and downloaded. Otherwise, the project definition manager will open
(see Chapter~\ref{chap:project_definitions_manager}) allowing the user to choose
one of the available project definitions. Once a project definition is
selected, the composite indicators delineated within the project definition are
re-calculated, and the layer is styled and rendered accordingly. This process
may take some time, depending on the complexity of the project.

\include{06_transform_attribute}
\include{07_project_definitions_manager}
\include{08_weighting_and_calculating}
\include{09_aggregate_loss_by_zone}
\chapter{Upload project to the OpenQuake Platform}

\begin{figure}
    \centering
    \includegraphics[width=\textwidth]{../images/image22}
    \caption{Simplified Integrated Risk analysis as it is seen inside QGIS
    right before the project is uploaded to the OpenQuake Platform}
    \label{fig:before_uploading}
\end{figure}

Once an integrated risk model is complete, and the user is satisfied with
results such as those obtained for the example displayed in
Figure~\ref{fig:before_uploading}, it is possible to upload projects through
the OQ-Platform. Projects are uploaded in order to share them with the wider
earthquake risk assessment, earthquake risk reduction, GIS communities, etc.
Uploading to the OQ-Platform also  supports the ability to visualize models
using advanced visualization tools and the mapping of the data over the web. In
addition, sharing the models on the OQ-Platform allows users that are not QGIS
savvy to dynamically interact with the data. The mapping and visualization over
the web is accomplished using the OQ-Platform
(Figure~\ref{fig:after_uploading}) and the Social Vulnerability and Integrated
Risk Viewer (the web application is available at
\url{http://www.globalquakemodel.org/openquake/support/documentation/platform/irv/}
and the corresponding documentation can be found at
\url{https://platform.openquake.org/irv_viewer/}).

\begin{figure}
    \centering
    \includegraphics[width=\textwidth]{../images/image10}
    \caption{The same simple example shown in
    Figure~\ref{fig:before_uploading}, visualized through a web browser after
    it has been uploaded to the OpenQuake Platform}
    \label{fig:after_uploading}
\end{figure}

To upload a project to the OQ-Platform, click `Upload project to the OpenQuake
Platform'. This will result in the opening of a dialog window in which,
depending on the context, the window will look like those delineated in
Figure~\ref{fig:upload_dialog} or~\ref{fig:update_dialog} 13 or 14. The former
will be displayed if the current project has never been uploaded to the
OQ-Platform. In such cases, the user is invited to provide a project title that
will become the title of the Geonode layer that will be created on the
Platform. A second field will contain the abstract, where the user can provide
a general description of the project.

\begin{figure}
    \centering
    \includegraphics[width=\textwidth]{../images/image25}
    \caption{Uploading a project to the OpenQuake Platform}
    \label{fig:upload_dialog}
\end{figure}

In order to be able to correctly utilize the advanced visualization tools found
on the OQ-Platform, the selection of a `Zone labels field' is required (see
Figure~\ref{fig:upload_dialog}). The user must designate the `Zone labels
field' within their dataset. The latter is a field containing unique labels (or
identifiers) whether these be individual country names, district names, or
census block numbers. Delineating a zone field when uploading to the
OQ-Platform is imperative to allow the graphing components of the Social
Vulnerability and Integrated Risk Viewer to render the visualization using the
zone's labels.  Without the latter, comparisons among places within the
graphing tools are not possible. It is also mandatory to choose a license and
to click on the checkbox to confirm to be informed about the license
conditions. By clicking the `Info' button, a web browser will be opened,
pointing to a page that describes the license selected in the `License'
dropdown menu. When `OK' is pressed, the active layer is uploaded to the
OQ-Platform and it is applied in the same style visible in QGIS\@. Furthermore,
the current project definition is saved into the Geonode layer's metadata,
inside the `Supplemental information' field.

\begin{figure}
    \centering
    \includegraphics[width=\textwidth]{../images/image21}
    \caption{Updating a project that has already been uploaded to the OpenQuake Platform}
    \label{fig:update_dialog}
\end{figure}

This second version of the `Upload' dialog window is displayed when the active
layer appears to have been already shared through the OQ-Platform (the ID of a
OQ-Platform's layer was previously associated with this layer). In such cases,
it is possible to create a brand new layer, ignoring the previously uploaded
(or downloaded) project, or to update the current layer. The updating process
consists of adding the current project definition to the set of project
definitions associated to that layer on the OQ-Platform. This is a much faster
procedure because no geometries need to be uploaded, and only the metadata of
the Geonode layer will be changed.


% %----------------------------------------------------------------------------------------
% %	CHAPTER 1
% %----------------------------------------------------------------------------------------
% \chapterimage{./figures/chapter_head_2.pdf} % Chapter heading image
% \chapter{Introduction}
% \label{chap:intro}
% \input{./mtk_setup.tex}

% %----------------------------------------------------------------------------------------
% %	CHAPTER 2
% %----------------------------------------------------------------------------------------
% \chapterimage{./figures/chapter_head_2.pdf} % Chapter heading image
% \chapter{Catalogue Tools}
% \label{chap:catalogue}
% \input{./catalogue_tools.tex}

% %----------------------------------------------------------------------------------------
% %	CHAPTER 3
% %----------------------------------------------------------------------------------------
% \chapterimage{./figures/chapter_head_1.pdf} % Chapter heading image
% \chapter{Hazard Tools}
% \label{chap:hazard}
% \input{./hazard_tools.tex}

% %----------------------------------------------------------------------------------------
% %	CHAPTER 4
% %----------------------------------------------------------------------------------------
% \chapterimage{./figures/chapter_head_1.pdf} % Chapter heading image
% \chapter{Geology Tools}
% \label{chap:geology}
% \input{./geology_tools.tex}

% %----------------------------------------------------------------------------------------
% %	CHAPTER 5
% %----------------------------------------------------------------------------------------
% \chapterimage{./figures/chapter_head_1.pdf} % Chapter heading image
% \chapter{Geodetic Tools}
% \label{chap:geodesy}
% \input{./geodetic_tools.tex}

% %----------------------------------------------------------------------------------------
% %	CHAPTER 6
% %----------------------------------------------------------------------------------------
% %\chapterimage{./figures/chapter_head_1.pdf} % Chapter heading image
% %\chapter{Hazard Applications}
% %\label{chap:hazard}
% %\input{./hazard.tex}


% %----------------------------------------------------------------------------------------
% %	BIBLIOGRAPHY
% %----------------------------------------------------------------------------------------
% \chapter*{Bibliography}
% \addcontentsline{toc}{chapter}{\textcolor{ocre}{Bibliography}}
% \section*{Books}
% \addcontentsline{toc}{section}{Books}
% \printbibliography[heading=bibempty,type=book]
% \section*{Articles}
% \addcontentsline{toc}{section}{Articles}
% \printbibliography[heading=bibempty,type=article]
% \section*{Other Sources}
% \addcontentsline{toc}{section}{Reports}
% \printbibliography[heading=bibempty,nottype=book,nottype=article]


% %----------------------------------------------------------------------------------------
% %	INDEX
% %----------------------------------------------------------------------------------------

% \cleardoublepage
% \phantomsection
% \setlength{\columnsep}{0.75cm}
% \addcontentsline{toc}{chapter}{\textcolor{ocre}{Index}}
% \printindex
% \printglossary

% %----------------------------------------------------------------------------------------


% \part{Appendices}
% \appendix
% % -----------------------------------------------------------------------------
% % -----------------------------------------------------------------------------
% \chapter{The 10 Minute Guide to Python!}
% \label{sec:python_guide}
% %\begin{myfancybox}
% % The objectives of this chapter are:
% %\begin{itemize}
% %     \item To introduce Python data types to facilitate use of the HMTK for Python beginners
% % \end{itemize}
% %\end{myfancybox}
% \input{python_guide.tex}


\end{document}
















%\documentclass[11pt,a4paper,headings=small,dvips]{scrbook}
%\setcounter{secnumdepth}{3}
\setcounter{tocdepth}{3}
% This is used to create the cover and to plot trees
\usepackage{pst-tree}
\usepackage{pstricks,pstricks-add,multido}
%
\usepackage{geometry}
\usepackage{moresize}
%
%\usepackage{algorithmic}
% 
\usepackage{fancyvrb}
\usepackage{listings}
\usepackage{alltt}
\usepackage{gensymb}
%
\usepackage[section]{placeins}
% 
\usepackage{pbox}
% http://en.wikibooks.org/wiki/LaTeX/Indexing
\usepackage{makeidx} 
\makeindex
%
%\usepackage{subfigure}
% Figure caption settings
\usepackage[textfont=it,margin=10pt,font=small,labelfont=bf,labelsep=endash]{caption}
\usepackage{subcaption}
%
\usepackage{bm}
% Landscape package
\usepackage{lscape}
%
\usepackage{hyperref} 
\hypersetup{colorlinks=true}
\hypersetup{breaklinks=true}
% package for multiline comments
\usepackage{verbatim}
%
% Package to create a glossary - It must be uploaded after hyperref
% to produce the glossary: makeglossaries OQB
%\usepackage[toc,acronym,nonumberlist,style=altlist]{glossaries}
\usepackage[toc,nonumberlist,style=altlist]{glossaries}
\glstoctrue
\makeglossaries
%
% - - - - - - - - - - - - - - - - - - - - - - - - - - - - - - - - Setting Fonts
% \renewcommand{\encodingdefault}{OT1}
\renewcommand{\encodingdefault}{OT1}
\renewcommand{\familydefault}{ppl}
% \renewcommand{\familydefault}{cmss}
% \renewcommand{\seriesdefault}{m}
% \renewcommand{\shapedefault}{up}

% - - - - - - - - - - - - - - - - - - - - - - - - - - - - - - - - - - - - - - -
\usepackage{amsmath}
% - - - - - - - - - - - - - - - - - - - - - - - - - - - - - - - - - - - - - - -
\usepackage{titlesec}
\usepackage[dvips]{graphicx}
% - - - - - - - - - - - - - - - - - - - - - - - - - - - - - - - - - - - - - - -
\usepackage{type1cm,eso-pic,color}

%\makeatletter
%\AddToShipoutPicture{
%    \setlength{\@tempdimb}{.5\paperwidth}
%    \setlength{\@tempdimc}{.5\paperheight}
%    \setlength{\unitlength}{1pt}
%    \put(\strip@pt\@tempdimb,\strip@pt\@tempdimc){
%        \makebox(0,0){\rotatebox{55}{
%        	\textcolor[gray]{0.85}{
%        		\fontsize{5cm}{5cm}
%        		\selectfont{DRAFT}}
%        	}
%        }
%	}
%}
%\makeatother

%
% Solves problems with margin notes
\usepackage{mparhack} 
	\setlength{\marginparwidth}{1.1in}
	\let\oldmarginpar\marginpar
	\renewcommand\marginpar[1]{\-\oldmarginpar[\raggedright\color{red01}
	\footnotesize #1]%
	{\raggedright\footnotesize #1}}
% Define some colors
	\definecolor{azure}{RGB}{240,255,255}
	\definecolor{honeydew}{RGB}{240,255,240}
	\definecolor{blue01}{RGB}{4,64,116}
	\definecolor{blue02}{RGB}{0,62,113}
	\definecolor{gray01}{rgb}{0.1,0.1,0.1}
	\definecolor{gray02}{rgb}{0.8,0.8,0.8}
	\definecolor{red01}{rgb}{0.5,0.0,0.0}
	\definecolor{orange00}{rgb}{1.0,0.74,0.53}
	\definecolor{orange01}{rgb}{0.9137,0.5882,0.0980}
	\definecolor{orange02}{rgb}{0.7608,0.4157,0.1804}
	\definecolor{orange03}{rgb}{0.6941,0.1843,0.1333}
\usepackage[english]{babel}
% Bibliography settings
\usepackage[square,colon]{natbib} % Extend bibligraphy functions
% Page numbering by Chapter
%\usepackage[auto]{chappg} 
%\pagenumbering{bychapter}
% 
% Define page properties
\usepackage{scrpage2}
	\pagestyle{scrheadings}
	\lofoot[]{\includegraphics[width=2.0cm]{./figures/openquake_logo1.eps}}
	\refoot[]{\includegraphics[width=2.0cm]{./figures/openquake_logo1.eps}}
	%\renewcommand{\partpagestyle}{empty}
% - - - - - - - - - - - - - - - - - - - - - - - - - -  Reformatting PART Titles
\titleformat{\part}[display]
{\filleft\normalfont\sffamily}
{\textcolor{blue01}{\bfseries\large PART}\hspace{4pt}
	\bfseries\Huge\textcolor{blue01}{\thepart}}
{1pc}
{\Huge\bfseries\textcolor{blue01}}
[]
% - - - - - - - - - - - - - - - - - - - - - - - - - Reformatting CHAPTER Titles
% Titles: CHAPTER
\titleformat{\chapter}
	[display] % shape
	{\filleft\normalfont\sffamily} % format
	{\textcolor{blue01}{\bfseries\MakeUppercase{\chaptertitlename}} % label
	\hspace{4pt}\huge\bfseries\textcolor{blue01}{\thechapter}} 
	{1pc} % sep
	{\huge\bfseries\textcolor{blue01}} % Before
	[]
% - - - - - - - - - - - - - - - - - - - - - - - - - Reformatting SECTION Titles
% Titles: SECTION
\titleformat{\section}
	[hang] % shape
	{\vspace{.8ex}\Large\bfseries\color{blue01}} % format 
	{\textcolor{blue01}{\thesection.}} % label
	{.5em} % sep
	{} % before
	[] % after
% - - - - - - - - - - - - - - - - - - - - - - -  Reformatting SUBSECTION Titles
% Title: SUBSECTION
\titleformat{\subsection}
	[hang] % shape
	{\vspace{.8ex}\large\bfseries\color{blue01}} % format 
	{\textcolor{blue01}{\thesubsection.}} % label
	{.5em} % sep
	{} % before
	[] % after
%  - - - - - - - - - - - - - - - - - - - - -  Reformatting SUBSUBSECTION Titles 
% Title: SUBSUBSECTION
\titleformat{\subsubsection}
	[hang] % shape
	{\vspace{.8ex}\normalfont\bfseries\color{blue01}} % format 
	{\textcolor{blue01}{\thesubsubsection.}} % label
	{.5em} % sep
	{} % before
	[] % after
% - - - - - - - - - - - - - - - - - - - - - - -  Reformatting PARAGRAPH Titles 
% Title: PARAGRAPH
\titleformat{\paragraph}
	[hang] % shape
	{\vspace{.2ex}\normalfont\color{blue01}} % format 
	{} % label
	{} % sep
	{} % before
	[] % after
%

%\usepackage{xcolor}
%\usepackage{framed}
%\usepackage[utf8]{inputenc}
%\usepackage{listings}
%
%\newenvironment{myfancybox}{%
%  \def\FrameCommand{\fboxsep=\FrameSep \fcolorbox{blue01}{honeydew}}%
%  \color{black}\MakeFramed {\FrameRestore}}%
% {\endMakeFramed}
%
%\setlength{\parskip}{2.5mm}
%\setlength{\parindent}{0.0mm}
%
%\begin{document}
%\setcounter{page}{1}
%\lstset{language=Python}
%
%\begin{titlepage}
%	\title{ \textcolor{blue01}{\textsf{\bfseries\Huge 
%        Hazard Modeller's Toolkit
%        }}}
%	\subtitle{ \textcolor{blue01}{\textsf{\bfseries\LARGE
%        Documentation \& Tutorial}}}
%	\date{June 2014}
% 
%	\publishers{GEM Foundation, Pavia}
%\end{titlepage}
%\pagestyle{scrheadings}
%\maketitle
%% - - - - - - - - - - - - - - - - - - - - - - - - - - - - - -  Load the glossary
%%\input{glossary.tex}
%% -----------------------------------------------------------------------------
%% -----------------------------------------------------------------------------
%%\chapter*{Introduction}
%%\cleardoublepage
%% -----------------------------------------------------------------------------
%% -----------------------------------------------------------------------------
%\tableofcontents
%\cleardoublepage
%% 
%% % -----------------------------------------------------------------------------
%\chapter{Introduction to the Hazard Modeller's Toolkit}
%\begin{myfancybox}
%The objectives of this chapter are:
%\begin{itemize}
%    \item Outline the purpose and creation of the Hazard Modeller's Toolkit 
%    \item Setup and installation of the software
%    \item Introduction to the visualisation and mapping functions
%\end{itemize}
%\end{myfancybox}
%  \input{mtk_setup.tex}
%\cleardoublepage
%% % -----------------------------------------------------------------------------
%% % -----------------------------------------------------------------------------
%
%\chapter{Seismicity Tools}
%\begin{myfancybox}
%The objectives of this chapter are:
%\begin{itemize}
%    \item Describe features of the seismicity tools of the Hazard Modeller's Toolkit 
%    \item Run simple calculations using the seismicity tools
%\end{itemize}
%\end{myfancybox}
%  \input{catalogue_tools.tex}
%\cleardoublepage
%
%
%% \cleardoublepage
%% %
%\chapter{Hazard Tools}
%\begin{myfancybox}
%The objectives of this chapter are:
%\begin{itemize}
%    \item Introduction to the HMTK source model classes
%    \item Understand how to link source models to seismicity
%    \item Run simple PSHA calculations in OpenQuake, using the HMTK
%\end{itemize}
%\end{myfancybox}
%  \input{hazard_tools.tex}
%\cleardoublepage
% 
%
%-----------------------------------------------------------------------------
%\chapter{Geological Tools}
%\begin{myfancybox}
%The objectives of this chapter are:
%\begin{itemize}
%    \item Describe features of the geological tools Hazard Modeller's Toolkit 
%    \item Run simple calculations using the geological tools 
%\end{itemize}
%\end{myfancybox}
%  \input{geology_tools.tex}
%\cleardoublepage
%
%
%% \cleardoublepage
%% % -----------------------------------------------------------------------------
%\chapter{Geodetic Tools}
%\begin{myfancybox}
%The objectives of this chapter are:
%\begin{itemize}
%    \item Describe features of the geodetic tools Hazard Modeller's Toolkit 
%    \item Run simple calculations using the geodetic tools 
%\end{itemize}
%\end{myfancybox}
%  \input{geodetic_tools.tex}
%\cleardoublepage


% -----------------------------------------------------------------------------

% -----------------------------------------------------------------------------
% -----------------------------------------------------------------------------
%%\cleardoublepage
%\bibliographystyle{apalike}
%\bibliography{./bibliography/hazard}
%\cleardoublepage
%\printglossaries
%\printindex
%% -----------------------------------------------------------------------------
% --------------------------------------------------------------------------
