\section{Introduction}
At the core of the Global Earthquake Model (GEM) is the development of
state-of-the-art modeling capabilities and a suite of software tools that can
be utilized worldwide for the assessment and communication of earthquake risk.
For a more holistic assessment of the scale and consequences of earthquake
impacts, a set of methods, metrics, and tools are incorporated into the GEM
modelling framework to assess earthquake impact potential beyond direct
physical impacts and loss of life. This is because with increased exposure of
people, livelihoods, and property to earthquakes, the potential for social and
economic impacts of earthquakes cannot be ignored. Not only is it vital to
evaluate and benchmark the conditions within social systems that lead to
adverse earthquake impacts and loss, it is equally important to measure the
capacity of populations to respond to damaging events and to provide a set of
metrics for priority setting and decision-making.

The employment  of a methodology and workflow necessary for the evaluation of
seismic risk that is integrated and holistic begins with the Integrated Risk
Modelling Toolkit (IRMT). The IRMT is QGIS plugin that was developed by the
\href{www.globalquakemodel.org}{Global Earthquake Model (GEM) Foundation} and
co-designed by GEM and the \href{www.cedim.de/english/index.php}{Center for
Disaster Management and Risk Reduction Technology (CEDIM)}. The plugin allows
users to form an integrated workflow for the construction of metrics used to
assess characteristics within societies that affect earthquake risk by
providing a GIS-based platform for the construction of indicators and composite
indices to foster comparative assessments. Here, an indicator is defined as a
piece of information that summarizes the characteristics of a system or
highlights what is happening in a system. An indicator is a quantitative or
qualitative measure derived from observed facts that simplify and communicate
the reality of a complex situation. Indicators reveal the relative position of
the phenomena being measured and when evaluated over time, can illustrate the
magnitude of change (a little or a lot) as well as direction of change (up or
down; increasing or decreasing). The mathematical combination (or aggregation
as it is termed) of a set of indicators forms a composite indicator (or
composite index or indices).

As part of the workflow, the IRMT facilitates the integration of composite
indicators of socio-economic characteristics with measures of physical risk
(i.e.\ estimations of human or economic loss) from the OpenQuake Engine
(OQ-engine)\footnote{Silva, V., Crowley, H., Pagani, M., Monelli, D., and
Pinho, R., 2014. Development of the OpenQuake engine, the Global Earthquake
Model's open-source software for seismic risk assessment. Natural Hazards
72(3), 1409-1427}, or other sources, to form what is referred to as an
integrated risk assessment. Although the tool may be utilized for any type of
indicator development, it is encouraged that composite indicators of social
vulnerability are developed within this integrated risk framework. Social
vulnerability is defined here as characteristics or qualities within social
systems that create the potential for harm or loss from damaging hazard events.
Given equal exposure to natural threats, such as an earthquake, two groups may
vary in their social vulnerability due to their pre-existing social
characteristics, where differences according to wealth, gender, race, class,
history, and sociopolitical organization influence the patterns of loss,
mortality, and the ability to reconstruct following damaging events.

The focus on the development of indicators of social vulnerability, and
ultimately integrated risk, will allow researchers, decision-makers, and other
relevant stakeholders to:

\begin{itemize}
    \item consider loss and damage as part of a dynamic system in which
        interactions between natural systems and societal factors redistribute
        risk before an event and redistribute loss after an event
    \item mainstream socio-economic vulnerability and resilience in earthquake
        loss and damage policy discussions
    \item evaluate loss and damage taking social factors into account at
        different time and space scales
    \item use risk assessments in benchmarking exercises to monitor trends in
        earthquake risk over time
    \item recognize that both causes and solutions for earthquake loss are
        found in human, environmental, and built-environmental interactions
    \item help decision-makers develop a common dialog that pertains to the
        factors that they should concentrate on to reduce risk and strengthen
        resilience.
\end{itemize}

The development of composite indicators is not new to research fields and
occupations requiring empirical measurement, and a vast literature on composite
indicators exists that outline methodological approaches for index construction
and validation. To accompany this manual we suggest the use of two popular
resources aimed at providing a guide for the construction and use of composite
indicators.

\begin{enumerate}
    \item Nardo, M., Saisana, M., Saltelli, A. and Tarantola, S. 2005. Tools
        for composite indicators Building. Ispara, Italy: Joint Research Center
        of the European Commission.
    \item Nardo, M., Saisana, M., Saltelli, A. and Tarantola, S. 2008. Handbook
        on constructing composite indicators: Methodology and user guide.
        Paris, France: OECD Publishing.
\end{enumerate}

This literature outlines the process of robust composite indicator construction
that contains a number of steps. The IRMT leverages the QGIS platform to guide
the user through the major steps for index construction. These steps include 1)
the selection of variables; 2) data normalization/standardization; 3) weighting
and aggregation to produce composite indicators; 4) risk integration using
OpenQuake risk estimates; and 5) the presentation of the results. Brief
descriptions of the tool's components and the workflow to develop integrated
risk models are outlined in the sections below.
