\section{What do we mean with the terms `layer', `project' and `project definition'?}
\label{sec:definitions}

In this manual, the terminology `layer', `project', and `project definition'
are used ubiquitously, and it is important to explain what the terminology
means as well as its use. In QGIS, a `project' or `project file' is a kind of
container that acts like a folder storing information on file locations of
layers and how these layers are displayed in a map. It is the main QGIS
datafile. A `layer' is the mechanism used to display geographic datasets in the
QGIS software, and layers provide the data that is manipulated within the
IRMT\@.  Each layer references a specific dataset and specifies how that
dataset is portrayed within the map. The standard layer format for the IRMT is
the ESRI Shapefile\footnote{ESRI 1998. ESRI Shapefile Technical Description,
Environmental Systems Research Institute, Redlands, C.A.} which can be imported
within the QGIS software using the default `add data' functionality, or layers
may be created on-the-fly within the IRMT using GEM's socio-economic databases.
A QGIS project can include multiple layers that can be utilized to provide the
variables and maps necessary for an integrated risk assessment. For each layer,
multiple `project definitions' can be saved. A `project definition' is a
tree-shaped model that is created within the IRMT to define the integrated risk
assessment's workflow. The project definition allows users to create, edit, and
manage the workflow needed to systematically develop integrated risk models
using layers. The project definition:

\begin{itemize}
    \item distinguishes which variables within a dataset are to be combined
        together to obtain a composite indicator;
    \item defines how variables are grouped together by supporting: 1)
        deductive models that typically contain fewer than ten indicators that
        are normalized and aggregated to create the index;  and 2) hierarchical
        models that employ roughly ten to twenty indicators that are separated
        into groups (sub-indices) that share the same underlying dimension
        (such as economy and infrastructure) in a manner in which individual
        indicators are aggregated into sub-indices, and the subindices are
        aggregated to create the index;
    \item describes  the type of aggregation method including additive
        modelling, weighted aggregation, and geometric aggregation that can be
        utilized by users to combine variables;
    \item establishes the application of weights (if desired) to individual
        variables or sub-indices; and
    \item delimits the directionality of variables when the intent is to
        consider that some variables may add to an index outcome; whereas some
        variables may may need to detract from it. When considering the social
        vulnerability of populations, a socio-economic status indicator such as
        the percentage of population with a college education provides an
        example of a characteristic that may detract from social vulnerability,
        thereby warranting a negative directionality within an index.
\end{itemize}
